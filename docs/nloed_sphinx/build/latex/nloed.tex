%% Generated by Sphinx.
\def\sphinxdocclass{report}
\documentclass[letterpaper,10pt,english,openany,oneside]{sphinxmanual}
\ifdefined\pdfpxdimen
   \let\sphinxpxdimen\pdfpxdimen\else\newdimen\sphinxpxdimen
\fi \sphinxpxdimen=.75bp\relax

\PassOptionsToPackage{warn}{textcomp}
\usepackage[utf8]{inputenc}
\ifdefined\DeclareUnicodeCharacter
% support both utf8 and utf8x syntaxes
  \ifdefined\DeclareUnicodeCharacterAsOptional
    \def\sphinxDUC#1{\DeclareUnicodeCharacter{"#1}}
  \else
    \let\sphinxDUC\DeclareUnicodeCharacter
  \fi
  \sphinxDUC{00A0}{\nobreakspace}
  \sphinxDUC{2500}{\sphinxunichar{2500}}
  \sphinxDUC{2502}{\sphinxunichar{2502}}
  \sphinxDUC{2514}{\sphinxunichar{2514}}
  \sphinxDUC{251C}{\sphinxunichar{251C}}
  \sphinxDUC{2572}{\textbackslash}
\fi
\usepackage{cmap}
\usepackage[T1]{fontenc}
\usepackage{amsmath,amssymb,amstext}
\usepackage{babel}



\usepackage{times}
\expandafter\ifx\csname T@LGR\endcsname\relax
\else
% LGR was declared as font encoding
  \substitutefont{LGR}{\rmdefault}{cmr}
  \substitutefont{LGR}{\sfdefault}{cmss}
  \substitutefont{LGR}{\ttdefault}{cmtt}
\fi
\expandafter\ifx\csname T@X2\endcsname\relax
  \expandafter\ifx\csname T@T2A\endcsname\relax
  \else
  % T2A was declared as font encoding
    \substitutefont{T2A}{\rmdefault}{cmr}
    \substitutefont{T2A}{\sfdefault}{cmss}
    \substitutefont{T2A}{\ttdefault}{cmtt}
  \fi
\else
% X2 was declared as font encoding
  \substitutefont{X2}{\rmdefault}{cmr}
  \substitutefont{X2}{\sfdefault}{cmss}
  \substitutefont{X2}{\ttdefault}{cmtt}
\fi


\usepackage[Bjarne]{fncychap}
\usepackage{sphinx}

\fvset{fontsize=\small}
\usepackage{geometry}


% Include hyperref last.
\usepackage{hyperref}
% Fix anchor placement for figures with captions.
\usepackage{hypcap}% it must be loaded after hyperref.
% Set up styles of URL: it should be placed after hyperref.
\urlstyle{same}

\addto\captionsenglish{\renewcommand{\contentsname}{Contents:}}

\usepackage{sphinxmessages}
\setcounter{tocdepth}{2}



\title{NLoed}
\date{Jan 19, 2021}
\release{0.0.1}
\author{Nathan Braniff}
\newcommand{\sphinxlogo}{\vbox{}}
\renewcommand{\releasename}{Release}
\makeindex
\begin{document}

\pagestyle{empty}
\sphinxmaketitle
\pagestyle{plain}
\sphinxtableofcontents
\pagestyle{normal}
\phantomsection\label{\detokenize{index::doc}}



\chapter{NLoed’s Model Class}
\label{\detokenize{nloed:module-nloed.model}}\label{\detokenize{nloed:nloed-s-model-class}}\label{\detokenize{nloed::doc}}\index{module@\spxentry{module}!nloed.model@\spxentry{nloed.model}}\index{nloed.model@\spxentry{nloed.model}!module@\spxentry{module}}\index{Model (class in nloed.model)@\spxentry{Model}\spxextra{class in nloed.model}}

\begin{fulllineitems}
\phantomsection\label{\detokenize{nloed:nloed.model.Model}}\pysiglinewithargsret{\sphinxbfcode{\sphinxupquote{class }}\sphinxcode{\sphinxupquote{nloed.model.}}\sphinxbfcode{\sphinxupquote{Model}}}{\emph{\DUrole{n}{observ\_struct}}, \emph{\DUrole{n}{input\_names}}, \emph{\DUrole{n}{param\_names}}, \emph{\DUrole{n}{options}\DUrole{o}{=}\DUrole{default_value}{\{\}}}}{}
Bases: \sphinxcode{\sphinxupquote{object}}

The NLoed Model class implements a mathematical model and provides useful functions for model building.

This class encodes casadi symbolic structures connecting model inputs and parameters to the
observation variables. The class also encodes the assumed distirbution for each model observation
variable.

Upon construction, the class populates a series of casadi function attributes for
computing various model predictions (i.e. mean, variance, sensitivity), logliklhoods, data
sampling, and the fisher information. These function attributes are used to implement the class’s
public user\sphinxhyphen{}callable functions; fit(), predict(), evaluate() and sample(). The function attributes
are also used by NLoed’s Design class when a Model instance is passsed for experimental design.
\index{symbolics\_boolean (nloed.model.Model attribute)@\spxentry{symbolics\_boolean}\spxextra{nloed.model.Model attribute}}

\begin{fulllineitems}
\phantomsection\label{\detokenize{nloed:nloed.model.Model.symbolics_boolean}}\pysigline{\sphinxbfcode{\sphinxupquote{symbolics\_boolean}}}
A boolean indicating if Casadi SX (True) or MX (false) symbolics
should be used.
\begin{quote}\begin{description}
\item[{Type}] \leavevmode
bool

\end{description}\end{quote}

\end{fulllineitems}

\index{num\_observ (nloed.model.Model attribute)@\spxentry{num\_observ}\spxextra{nloed.model.Model attribute}}

\begin{fulllineitems}
\phantomsection\label{\detokenize{nloed:nloed.model.Model.num_observ}}\pysigline{\sphinxbfcode{\sphinxupquote{num\_observ}}}
An integer indicating the number of observation variables in the model.
\begin{quote}\begin{description}
\item[{Type}] \leavevmode
integer

\end{description}\end{quote}

\end{fulllineitems}

\index{num\_input (nloed.model.Model attribute)@\spxentry{num\_input}\spxextra{nloed.model.Model attribute}}

\begin{fulllineitems}
\phantomsection\label{\detokenize{nloed:nloed.model.Model.num_input}}\pysigline{\sphinxbfcode{\sphinxupquote{num\_input}}}
An integer indicating the number of input variables accepted by the model.
\begin{quote}\begin{description}
\item[{Type}] \leavevmode
integer

\end{description}\end{quote}

\end{fulllineitems}

\index{num\_param (nloed.model.Model attribute)@\spxentry{num\_param}\spxextra{nloed.model.Model attribute}}

\begin{fulllineitems}
\phantomsection\label{\detokenize{nloed:nloed.model.Model.num_param}}\pysigline{\sphinxbfcode{\sphinxupquote{num\_param}}}
An integer indicating the number of parameters accepted by the model.
\begin{quote}\begin{description}
\item[{Type}] \leavevmode
integer

\end{description}\end{quote}

\end{fulllineitems}

\index{input\_name\_list (nloed.model.Model attribute)@\spxentry{input\_name\_list}\spxextra{nloed.model.Model attribute}}

\begin{fulllineitems}
\phantomsection\label{\detokenize{nloed:nloed.model.Model.input_name_list}}\pysigline{\sphinxbfcode{\sphinxupquote{input\_name\_list}}}
A list of the input variable names, in the order passed
to the model constructor.
\begin{quote}\begin{description}
\item[{Type}] \leavevmode
list of strings

\end{description}\end{quote}

\end{fulllineitems}

\index{param\_name\_list (nloed.model.Model attribute)@\spxentry{param\_name\_list}\spxextra{nloed.model.Model attribute}}

\begin{fulllineitems}
\phantomsection\label{\detokenize{nloed:nloed.model.Model.param_name_list}}\pysigline{\sphinxbfcode{\sphinxupquote{param\_name\_list}}}
A list of the parameter names, in the order passed
to the model constructor.
\begin{quote}\begin{description}
\item[{Type}] \leavevmode
list of strings

\end{description}\end{quote}

\end{fulllineitems}

\index{observ\_name\_list (nloed.model.Model attribute)@\spxentry{observ\_name\_list}\spxextra{nloed.model.Model attribute}}

\begin{fulllineitems}
\phantomsection\label{\detokenize{nloed:nloed.model.Model.observ_name_list}}\pysigline{\sphinxbfcode{\sphinxupquote{observ\_name\_list}}}
A list of the observation variable names, in the order
passed to the model constructor.
\begin{quote}\begin{description}
\item[{Type}] \leavevmode
list of strings

\end{description}\end{quote}

\end{fulllineitems}

\index{loglik (nloed.model.Model attribute)@\spxentry{loglik}\spxextra{nloed.model.Model attribute}}

\begin{fulllineitems}
\phantomsection\label{\detokenize{nloed:nloed.model.Model.loglik}}\pysigline{\sphinxbfcode{\sphinxupquote{loglik}}}
This function attribute consists of a dictionary in which
the keys are the model’s observation variable names and the values are casadi functions
computing the loglikelihood of a single observation of the variable at the passed
observation value with the passed input and parameter settings.

Call Structure: Model.loglik{[}obs\_name{]}(obs\_value, inputs, parameters)
\begin{quote}\begin{description}
\item[{Type}] \leavevmode
dictionary of functions

\end{description}\end{quote}

\end{fulllineitems}

\index{fisher\_info\_matrix (nloed.model.Model attribute)@\spxentry{fisher\_info\_matrix}\spxextra{nloed.model.Model attribute}}

\begin{fulllineitems}
\phantomsection\label{\detokenize{nloed:nloed.model.Model.fisher_info_matrix}}\pysigline{\sphinxbfcode{\sphinxupquote{fisher\_info\_matrix}}}
This function attribute consists of a
dictionary in which the keys are the model’s observation variable names and the values
are casadi functions computing the fisher information matrix for a single observation of
the specified observation variable at the passed input and parameter settings.
Model.model\_mean(inputs, parameters)

Call Structure: Model.fisher\_info\_matrix{[}obs\_name{]}(inputs, parameters)
\begin{quote}\begin{description}
\item[{Type}] \leavevmode
dictionary of functions

\end{description}\end{quote}

\end{fulllineitems}

\index{model\_mean (nloed.model.Model attribute)@\spxentry{model\_mean}\spxextra{nloed.model.Model attribute}}

\begin{fulllineitems}
\phantomsection\label{\detokenize{nloed:nloed.model.Model.model_mean}}\pysigline{\sphinxbfcode{\sphinxupquote{model\_mean}}}
This function attribute consists of a
dictionary in which the keys are the model’s observation variable names and the values
are casadi functions computing the expected mean observation value at the passed input
and parameter settings.

Call Structure: Model.model\_mean{[}obs\_name{]}(inputs, parameters)
\begin{quote}\begin{description}
\item[{Type}] \leavevmode
dictionary of functions

\end{description}\end{quote}

\end{fulllineitems}

\index{model\_variance (nloed.model.Model attribute)@\spxentry{model\_variance}\spxextra{nloed.model.Model attribute}}

\begin{fulllineitems}
\phantomsection\label{\detokenize{nloed:nloed.model.Model.model_variance}}\pysigline{\sphinxbfcode{\sphinxupquote{model\_variance}}}
This function attribute consists of a
dictionary in which the keys are the model’s observation variable names and the values
are casadi functions computing the expected variance of the observation variable at the
passed input and parameter settings.

Call Structure: Model.model\_variance{[}obs\_name{]}(inputs, parameters)
\begin{quote}\begin{description}
\item[{Type}] \leavevmode
dictionary of functions

\end{description}\end{quote}

\end{fulllineitems}

\index{model\_sensitivity (nloed.model.Model attribute)@\spxentry{model\_sensitivity}\spxextra{nloed.model.Model attribute}}

\begin{fulllineitems}
\phantomsection\label{\detokenize{nloed:nloed.model.Model.model_sensitivity}}\pysigline{\sphinxbfcode{\sphinxupquote{model\_sensitivity}}}
This function attribute consists of a
dictionary in which the keys are the model’s observation variable names and the values
are casadi functions computing the parameteric sensitivity of the expected mean
observation of the specified variable at the passed input and parameter settings.

Call Structure: Model.model\_variance{[}obs\_name{]}(inputs, parameters)
\begin{quote}\begin{description}
\item[{Type}] \leavevmode
dictionary of functions

\end{description}\end{quote}

\end{fulllineitems}

\index{observation\_sampler (nloed.model.Model attribute)@\spxentry{observation\_sampler}\spxextra{nloed.model.Model attribute}}

\begin{fulllineitems}
\phantomsection\label{\detokenize{nloed:nloed.model.Model.observation_sampler}}\pysigline{\sphinxbfcode{\sphinxupquote{observation\_sampler}}}
This function attribute consists of a
dictionary in which the keys are the model’s observation variable names and the values
are casadi\sphinxhyphen{}based functions generating random realizations of the observation variable value
from its expected distribution, conditioned on the passed input and parameter settings.

Call Structure: Model.model\_variance{[}obs\_name{]}(inputs, parameters)
\begin{quote}\begin{description}
\item[{Type}] \leavevmode
dictionary of functions

\end{description}\end{quote}

\end{fulllineitems}

\index{observation\_percentile (nloed.model.Model attribute)@\spxentry{observation\_percentile}\spxextra{nloed.model.Model attribute}}

\begin{fulllineitems}
\phantomsection\label{\detokenize{nloed:nloed.model.Model.observation_percentile}}\pysigline{\sphinxbfcode{\sphinxupquote{observation\_percentile}}}
This function attribute consists of a
dictionary in which the keys are the model’s observation variable names and the values
are casadi\sphinxhyphen{}based functions computing the requested percentile of the observation variable’s
distribution conditioned on the passed input and parameter settings.

Call Structure: Model.model\_variance{[}obs\_name{]}(percentile, inputs, parameters)
\begin{quote}\begin{description}
\item[{Type}] \leavevmode
dictionary of functions

\end{description}\end{quote}

\end{fulllineitems}

\index{distribution\_dict (nloed.model.Model attribute)@\spxentry{distribution\_dict}\spxextra{nloed.model.Model attribute}}

\begin{fulllineitems}
\phantomsection\label{\detokenize{nloed:nloed.model.Model.distribution_dict}}\pysigline{\sphinxbfcode{\sphinxupquote{distribution\_dict}}\sphinxbfcode{\sphinxupquote{ = \{\textquotesingle{}Bernoulli\textquotesingle{}: {[}\textquotesingle{}Probability\textquotesingle{}{]}, \textquotesingle{}Binomial\textquotesingle{}: {[}\textquotesingle{}Probability\textquotesingle{}{]}, \textquotesingle{}Exponential\textquotesingle{}: {[}\textquotesingle{}Rate\textquotesingle{}{]}, \textquotesingle{}Gamma\textquotesingle{}: {[}\textquotesingle{}Shape\textquotesingle{}, \textquotesingle{}Scale\textquotesingle{}{]}, \textquotesingle{}Lognormal\textquotesingle{}: {[}\textquotesingle{}GeoMean\textquotesingle{}, \textquotesingle{}GeoVariance\textquotesingle{}{]}, \textquotesingle{}Normal\textquotesingle{}: {[}\textquotesingle{}Mean\textquotesingle{}, \textquotesingle{}Variance\textquotesingle{}{]}, \textquotesingle{}Poisson\textquotesingle{}: {[}\textquotesingle{}Rate\textquotesingle{}{]}\}}}}
\end{fulllineitems}

\index{fit() (nloed.model.Model method)@\spxentry{fit()}\spxextra{nloed.model.Model method}}

\begin{fulllineitems}
\phantomsection\label{\detokenize{nloed:nloed.model.Model.fit}}\pysiglinewithargsret{\sphinxbfcode{\sphinxupquote{fit}}}{\emph{\DUrole{n}{datasets}}, \emph{\DUrole{n}{start\_param}\DUrole{o}{=}\DUrole{default_value}{None}}, \emph{\DUrole{n}{options}\DUrole{o}{=}\DUrole{default_value}{\{\}}}}{}
A function for fitting the NLoed model to a dataset contained in a dataframe.

This function fits the model to a dataset using maximum likelihood. This function can also
return marginal confidence intervals as well as plots of liklihood profiles and projections
of profiles traces and confidence contours.
\begin{quote}\begin{description}
\item[{Parameters}] \leavevmode\begin{itemize}
\item {} 
\sphinxstyleliteralstrong{\sphinxupquote{datasets}} (\sphinxstyleliteralemphasis{\sphinxupquote{dataframe OR list of dataframes}}) \textendash{} A dataframe containing the dataset to be fit.
OR a list of dataframes, each containing a dataset replicate of a given design
OR a list of lists of dataframes, where each index in the outer list corresponds
to a unique design and each inner index coresponds to a replicate of the given design

\item {} 
\sphinxstyleliteralstrong{\sphinxupquote{start\_param}} (\sphinxstyleliteralemphasis{\sphinxupquote{array\sphinxhyphen{}like}}\sphinxstyleliteralemphasis{\sphinxupquote{, }}\sphinxstyleliteralemphasis{\sphinxupquote{optional}}) \textendash{} An array of starting parameter values where the
local fitting optimization should be started

\item {} 
\sphinxstyleliteralstrong{\sphinxupquote{options}} (\sphinxstyleliteralemphasis{\sphinxupquote{dict}}\sphinxstyleliteralemphasis{\sphinxupquote{, }}\sphinxstyleliteralemphasis{\sphinxupquote{optional}}) \textendash{} 
A dictionary of user\sphinxhyphen{}defined options, possible key\sphinxhyphen{}value pairs
include:

”Confidence” \textendash{}
Purpose: Determines confidnece diagnostics to be returned or plotted,
Type: string,
Default Value: “None”,
Possible Values:
“None” = No intervals returned,
“Intervals” = Marginal intervals returned,
“Profiles” = Same as “Intervals” but trace projections are plotted using Matplotlib,
“Contours” = Same as  “Profiles” but confidence contour projections are also plotted.

”ConfidenceLevel” \textendash{}
Purpose: Sets the confidence level for the marginal intervals, traces, profiles and contours,
Type: float,
Default Value: 0.95,
Possible Values: 0\textless{}1.

’RadialNumber’ \textendash{}
Purpose: Determines the number of radial searches performed out from the fit
parameter value used to find the confidence contour projections,
Type: integer
Default Value: 30
Possible Values: \textgreater{}1, but sufficient density is needed for good interpolation

”SampleNumber” \textendash{}
Purpose:
Type: integer
Default Value: 10
Possible Values: \textgreater{}1

”Tolerance” \textendash{}
Purpose:
Type: float
Default Value: 0.001
Possible Values: \textgreater{}0

”InitialStep” \textendash{}
Purpose:
Type: float
Default Value: 0.01
Possible Values: \textgreater{}0

”MaxSteps” \textendash{}
Purpose:
Type: integer
Default Value: 2000
Possible Values: \textgreater{}1

”SearchFactor” \textendash{}
Purpose:
Type: float
Default Value: 5.0
Possible Values: \textgreater{}0

”InitParamBounds” \textendash{}
Purpose:
Type: array\sphinxhyphen{}like
Default Value: False
Possible Values:

”InitSearchNumber” \textendash{}
Purpose:
Type: integer
Default Value: 3
Possible Values: \textgreater{}0

”Verbose” \textendash{}
Purpose:
Type: boolean
Default Value: True
Possible Values: True or False


\end{itemize}

\item[{Returns}] \leavevmode
A dataframe containing the fit parameters,
and if requested, confidence interval information. If a list of datasets was provided,
each row of the dataframe corresponds to the dataset index in the passed list.
If a list of lists of dataframes was provided (designs by replicates),
a list of dataframes will be returned with the same length as the outer index of the
passed list of lists.

\item[{Return type}] \leavevmode
dataframe OR list of dataframes

\end{description}\end{quote}

\end{fulllineitems}

\index{sample() (nloed.model.Model method)@\spxentry{sample()}\spxextra{nloed.model.Model method}}

\begin{fulllineitems}
\phantomsection\label{\detokenize{nloed:nloed.model.Model.sample}}\pysiglinewithargsret{\sphinxbfcode{\sphinxupquote{sample}}}{\emph{\DUrole{n}{designs}}, \emph{\DUrole{n}{param}}, \emph{\DUrole{n}{design\_replicates}\DUrole{o}{=}\DUrole{default_value}{1}}, \emph{\DUrole{n}{options}\DUrole{o}{=}\DUrole{default_value}{\{\}}}}{}
A function for generating simulated data from the NLoed model for a given design passed
as a dataframe.

This function generates datasets using the NLoed model and a provided design (or a list of
designs) via simulation and Numpy/Scipy’s random number generation. This simulation is done
at a user\sphinxhyphen{}provided set of parameter values. The number of replicates of the design that are
simulated is optional but defaults to one.
\begin{quote}\begin{description}
\item[{Parameters}] \leavevmode\begin{itemize}
\item {} 
\sphinxstyleliteralstrong{\sphinxupquote{designs}} (\sphinxstyleliteralemphasis{\sphinxupquote{dataframe OR list of dataframes}}) \textendash{} A dataframe containing the design to be simulated,
OR a list of dataframes containing multiple designs to be simulated

\item {} 
\sphinxstyleliteralstrong{\sphinxupquote{param}} (\sphinxstyleliteralemphasis{\sphinxupquote{array\sphinxhyphen{}like}}) \textendash{} The parameter values used to generate the data

\item {} 
\sphinxstyleliteralstrong{\sphinxupquote{design\_replicates}} (\sphinxstyleliteralemphasis{\sphinxupquote{integer}}\sphinxstyleliteralemphasis{\sphinxupquote{, }}\sphinxstyleliteralemphasis{\sphinxupquote{optional}}) \textendash{} An integer indicating the number of dataset
replicates to be generated for each design. The default is 1 per design passed.

\item {} 
\sphinxstyleliteralstrong{\sphinxupquote{options}} (\sphinxstyleliteralemphasis{\sphinxupquote{dict}}\sphinxstyleliteralemphasis{\sphinxupquote{, }}\sphinxstyleliteralemphasis{\sphinxupquote{optional}}) \textendash{} 
A dictionary of user\sphinxhyphen{}defined options, possible key\sphinxhyphen{}value pairs
include:

”Verbose” \textendash{}
Purpose: Determines the amount of print feedback provided while the function executes
Type: bool,
Default Value: True,
Possible Values: True or False


\end{itemize}

\item[{Returns}] \leavevmode
A dataframe containg a simulation of the design is returned by default
OR, if design\_replicates was set to \textgreater{}1, a list of dataframes containg simulated
replicates is returned for the given design
OR,  if a list of dataframes containing a set of design was passed,
a list of lists of dataframes is returned, the outer index corresponds to the design
list, and the inner index corresponds to the number of replicates requested.

\item[{Return type}] \leavevmode
dataframe OR list of dataframes

\end{description}\end{quote}

\end{fulllineitems}

\index{predict() (nloed.model.Model method)@\spxentry{predict()}\spxextra{nloed.model.Model method}}

\begin{fulllineitems}
\phantomsection\label{\detokenize{nloed:nloed.model.Model.predict}}\pysiglinewithargsret{\sphinxbfcode{\sphinxupquote{predict}}}{\emph{\DUrole{n}{input\_struct}}, \emph{\DUrole{n}{param}}, \emph{\DUrole{n}{covariance\_matrix}\DUrole{o}{=}\DUrole{default_value}{None}}, \emph{\DUrole{n}{options}\DUrole{o}{=}\DUrole{default_value}{\{\}}}}{}
A function for generating prediction information from the NLoed model.

This function can be used to compute predictions from an NLoed model instance given the
user\sphinxhyphen{}provided input conditions and paramater values. Prediction information includes the
predicted mean response of the model, confidence intervals for the mean response under
parameter uncertainty, confidence intervals for the observation distribution, and sensitivity
analysis of the mean response. The returned intervals can be computed in a number of ways;
exactly, with a normal (local and deterministic) approximation, or using Monte Carlo simulation.
\begin{quote}\begin{description}
\item[{Parameters}] \leavevmode\begin{itemize}
\item {} 
\sphinxstyleliteralstrong{\sphinxupquote{input\_struct}} (\sphinxstyleliteralemphasis{\sphinxupquote{dataframe}}) \textendash{} A dataframe specifying the combination of inputs values and
observations at which predictions are desired.

\item {} 
\sphinxstyleliteralstrong{\sphinxupquote{param}} (\sphinxstyleliteralemphasis{\sphinxupquote{array\sphinxhyphen{}like}}) \textendash{} The parameter vector values at which the predictions are to be made.

\item {} 
\sphinxstyleliteralstrong{\sphinxupquote{(}}\sphinxstyleliteralstrong{\sphinxupquote{array\sphinxhyphen{}like}} (\sphinxstyleliteralemphasis{\sphinxupquote{covariance\_matrix}}) \textendash{} A symetric matrix specifying the parameter’s
normal covariance matrix, required if parameter uncertainty is to be included. The
prior mean set by the values passed via the param argument.

\item {} 
\sphinxstyleliteralstrong{\sphinxupquote{optional}} \textendash{} A symetric matrix specifying the parameter’s
normal covariance matrix, required if parameter uncertainty is to be included. The
prior mean set by the values passed via the param argument.

\item {} 
\sphinxstyleliteralstrong{\sphinxupquote{options}} (\sphinxstyleliteralemphasis{\sphinxupquote{dictionary}}\sphinxstyleliteralemphasis{\sphinxupquote{, }}\sphinxstyleliteralemphasis{\sphinxupquote{optional}}) \textendash{} 
A dictionary of user\sphinxhyphen{}defined options, possible key\sphinxhyphen{}value
pairs include:

”Method” \textendash{}
Purpose: Selects the method used to compute the mean and prediction and observation
intervals.
Type: string
Default Value: “Delta”
Possible Values: “Exact”=Compute predictions and intervals exactly; this
option is only available if no parameter uncertainty information is NOT provided .
“Delta”=Use a normal, local apporximation to compute the prediction and
intervals, “MonteCarlo”=Use Monte Carlo simulation of the model to compute the
prediction and intervals under any parameter or observation uncertainty.

”PredictionInterval” \textendash{}
Purpose: A boolean to indicat if prediction intervals are to be returned, true if yes.
Type: bool
Default Value: False
Possible Values: True or False

”ObservationInterval” \textendash{}
Purpose: A boolean to indicat if observation intervals are to be returned, true if yes.
Type: bool
Default Value: False
Possible Values: True or False

”Sensitivity” \textendash{}
Purpose: A boolean to indicat if observation intervals are to be returned, true if yes.
Type: bool
Default Value: False
Possible Values: True or False

”PredictionSampleNumber” \textendash{}
Purpose: An integer indicating the number of parameter vectors to be sampled from
the prior in order to compute the prediction (and observation0 intervals using Monte
Carlo simulation.
Type: integer
Default Value: 10000
Possible Values:

”ObservationSampleNumber” \textendash{}
Purpose: An integer indicating the number of observation values to be sampled from
the observation distribution in order to compute the observation interval using
Monte Carlo simulation.
Type: integer
Default Value: 10
Possible Values:

”ConfidenceLevel” \textendash{}
Purpose: A float specifying the confidence level desired for any intervals computed.
Type: float
Default Value: 0.95
Possible Values: \textless{}1, \textgreater{}0


\end{itemize}

\item[{Returns}] \leavevmode
\begin{description}
\item[{A dataframe containing the requested prediction quntities computed at the}] \leavevmode
input and parameter settings passed.

\end{description}


\item[{Return type}] \leavevmode
dataframe

\end{description}\end{quote}

\end{fulllineitems}

\index{evaluate() (nloed.model.Model method)@\spxentry{evaluate()}\spxextra{nloed.model.Model method}}

\begin{fulllineitems}
\phantomsection\label{\detokenize{nloed:nloed.model.Model.evaluate}}\pysiglinewithargsret{\sphinxbfcode{\sphinxupquote{evaluate}}}{\emph{\DUrole{n}{designs}}, \emph{\DUrole{n}{param}}, \emph{\DUrole{n}{options}\DUrole{o}{=}\DUrole{default_value}{\{\}}}}{}
A function for evaluating the peformance metrics of a passed design with the NLoed Model.

This function

Args:

Return:

\end{fulllineitems}

\index{\_get\_distribution\_functions() (nloed.model.Model method)@\spxentry{\_get\_distribution\_functions()}\spxextra{nloed.model.Model method}}

\begin{fulllineitems}
\phantomsection\label{\detokenize{nloed:nloed.model.Model._get_distribution_functions}}\pysiglinewithargsret{\sphinxbfcode{\sphinxupquote{\_get\_distribution\_functions}}}{\emph{\DUrole{n}{observ\_model}}, \emph{\DUrole{n}{observ\_distribution}}}{}
A private function that automatically generates function attributes for the given
observation variable information.

This private function is used during the NLoed model class construction.

This function accepts the observation name, distribution type
and observation model and constructs casadi functions to compute
the logliklihood, fisher information, prediction mean/sensitivity/variance
and also a numpy/casadi function to return random samples from the the model
\begin{quote}\begin{description}
\item[{Parameters}] \leavevmode\begin{itemize}
\item {} 
\sphinxstyleliteralstrong{\sphinxupquote{observ\_name}} \textendash{} a string specifying the observation variable name

\item {} 
\sphinxstyleliteralstrong{\sphinxupquote{observ\_distribution}} \textendash{} a string specifying the observation distribution type (i.e ‘Normal’)

\item {} 
\sphinxstyleliteralstrong{\sphinxupquote{observ\_model}} \textendash{} a casadi function mapping input and parameter vectors to sampling distribution statistics

\end{itemize}

\item[{Returns}] \leavevmode
\begin{description}
\item[{A list of casadi/numpy functions in the following order;}] \leavevmode
loglik, fisher info, prediction mean , prediction mean sensitivity,
prediction variance, observation sampler

\end{description}


\item[{Return type}] \leavevmode
list

\end{description}\end{quote}

\end{fulllineitems}

\index{\_\_confidence\_intervals() (nloed.model.Model method)@\spxentry{\_\_confidence\_intervals()}\spxextra{nloed.model.Model method}}

\begin{fulllineitems}
\phantomsection\label{\detokenize{nloed:nloed.model.Model.__confidence_intervals}}\pysiglinewithargsret{\sphinxbfcode{\sphinxupquote{\_\_confidence\_intervals}}}{\emph{\DUrole{n}{mle\_params}}, \emph{\DUrole{n}{loglik\_func}}, \emph{\DUrole{n}{options}}}{}
This function computes marginal parameter confidence intervals for the model
around the MLE estimate using the profile likelihood
\begin{quote}\begin{description}
\item[{Parameters}] \leavevmode\begin{itemize}
\item {} 
\sphinxstyleliteralstrong{\sphinxupquote{mle\_params}} \textendash{} mle parameter estimates,  recieved from fitting

\item {} 
\sphinxstyleliteralstrong{\sphinxupquote{loglik\_func}} \textendash{} casadi logliklihood function for the given dataset

\item {} 
\sphinxstyleliteralstrong{\sphinxupquote{options}} \textendash{} an options dictionary for passing user options

\end{itemize}

\item[{Returns}] \leavevmode
list of lists of upper and lower bounds for each parameter

\item[{Return type}] \leavevmode
interval\_list

\end{description}\end{quote}

\end{fulllineitems}

\index{\_\_profileplot() (nloed.model.Model method)@\spxentry{\_\_profileplot()}\spxextra{nloed.model.Model method}}

\begin{fulllineitems}
\phantomsection\label{\detokenize{nloed:nloed.model.Model.__profileplot}}\pysiglinewithargsret{\sphinxbfcode{\sphinxupquote{\_\_profileplot}}}{\emph{\DUrole{n}{mle\_params}}, \emph{\DUrole{n}{loglik\_func}}, \emph{\DUrole{n}{figure}}, \emph{\DUrole{n}{options}}}{}
This function plots profile parameter traces for each parameter value
\begin{quote}\begin{description}
\item[{Parameters}] \leavevmode\begin{itemize}
\item {} 
\sphinxstyleliteralstrong{\sphinxupquote{mle\_params}} \textendash{} mle parameter estimates,  recieved from fitting

\item {} 
\sphinxstyleliteralstrong{\sphinxupquote{loglik\_func}} \textendash{} casadi logliklihood function for the given dataset

\item {} 
\sphinxstyleliteralstrong{\sphinxupquote{figure}} \textendash{} the figure object on which plotting occurs

\item {} 
\sphinxstyleliteralstrong{\sphinxupquote{options}} \textendash{} an options dictionary for passing user options

\end{itemize}

\item[{Returns}] \leavevmode
list of lists of upper and lower bounds for each parameter
trace\_list: list of list of lists of parameter vector values along profile trace for each parameter
profile\_list: List of lists of logliklihood ratio values for each parameter along the profile trace

\item[{Return type}] \leavevmode
interval\_list

\end{description}\end{quote}

\end{fulllineitems}

\index{\_\_profiletrace() (nloed.model.Model method)@\spxentry{\_\_profiletrace()}\spxextra{nloed.model.Model method}}

\begin{fulllineitems}
\phantomsection\label{\detokenize{nloed:nloed.model.Model.__profiletrace}}\pysiglinewithargsret{\sphinxbfcode{\sphinxupquote{\_\_profiletrace}}}{\emph{\DUrole{n}{mle\_params}}, \emph{\DUrole{n}{loglik\_func}}, \emph{\DUrole{n}{options}}}{}
This function compute the profile logliklihood parameter trace for each parameter in the model
\begin{quote}\begin{description}
\item[{Parameters}] \leavevmode\begin{itemize}
\item {} 
\sphinxstyleliteralstrong{\sphinxupquote{mle\_params}} \textendash{} mle parameter estimates,  recieved from fitting

\item {} 
\sphinxstyleliteralstrong{\sphinxupquote{loglik\_func}} \textendash{} casadi logliklihood function for the given dataset

\item {} 
\sphinxstyleliteralstrong{\sphinxupquote{options}} \textendash{} an options dictionary for passing user options

\end{itemize}

\item[{Returns}] \leavevmode
list of lists of upper and lower bounds for each parameter
trace\_list: list of list of lists of parameter vector values along profile trace for each parameter
profile\_list: List of lists of logliklihood ratio values for each parameter along the profile trace

\item[{Return type}] \leavevmode
interval\_list

\end{description}\end{quote}

\end{fulllineitems}

\index{\_\_contourplot() (nloed.model.Model method)@\spxentry{\_\_contourplot()}\spxextra{nloed.model.Model method}}

\begin{fulllineitems}
\phantomsection\label{\detokenize{nloed:nloed.model.Model.__contourplot}}\pysiglinewithargsret{\sphinxbfcode{\sphinxupquote{\_\_contourplot}}}{\emph{\DUrole{n}{mle\_params}}, \emph{\DUrole{n}{loglik\_func}}, \emph{\DUrole{n}{figure}}, \emph{\DUrole{n}{options}}}{}
This function plots the projections of the confidence volume in a 2d plane for each pair of parameters
this creates marginal confidence contours for each pair of parameters
\begin{quote}\begin{description}
\item[{Parameters}] \leavevmode\begin{itemize}
\item {} 
\sphinxstyleliteralstrong{\sphinxupquote{mle\_params}} \textendash{} mle parameter estimates,  recieved from fitting

\item {} 
\sphinxstyleliteralstrong{\sphinxupquote{loglik\_func}} \textendash{} casadi logliklihood function for the given dataset

\item {} 
\sphinxstyleliteralstrong{\sphinxupquote{figure}} \textendash{} the figure object on which plotting occurs

\item {} 
\sphinxstyleliteralstrong{\sphinxupquote{options}} \textendash{} an options dictionary for passing user options

\end{itemize}

\end{description}\end{quote}

\end{fulllineitems}

\index{\_\_contourtrace() (nloed.model.Model method)@\spxentry{\_\_contourtrace()}\spxextra{nloed.model.Model method}}

\begin{fulllineitems}
\phantomsection\label{\detokenize{nloed:nloed.model.Model.__contourtrace}}\pysiglinewithargsret{\sphinxbfcode{\sphinxupquote{\_\_contourtrace}}}{\emph{\DUrole{n}{mle\_params}}, \emph{\DUrole{n}{loglik\_func}}, \emph{\DUrole{n}{coordinates}}, \emph{\DUrole{n}{options}}}{}
This function plots the projections of the confidence volume in a 2d plane for each pair of parameters
this creates marginal confidence contours for each pair of parameters
\begin{quote}\begin{description}
\item[{Parameters}] \leavevmode\begin{itemize}
\item {} 
\sphinxstyleliteralstrong{\sphinxupquote{mle\_params}} \textendash{} mle parameter estimates,  recieved from fitting

\item {} 
\sphinxstyleliteralstrong{\sphinxupquote{loglik\_func}} \textendash{} casadi logliklihood function for the given dataset

\item {} 
\sphinxstyleliteralstrong{\sphinxupquote{coordinates}} \textendash{} a pair of parameter coordinates specifying the 2d contour to be computed in parameter space

\item {} 
\sphinxstyleliteralstrong{\sphinxupquote{options}} \textendash{} an options dictionary for passing user options

\end{itemize}

\item[{Returns}] \leavevmode
x, y\sphinxhyphen{}values in parameter space specified by coordinates tracing the projected profile confidence contour outline

\item[{Return type}] \leavevmode
{[}x\_fit, y\_fit{]}

\end{description}\end{quote}

\end{fulllineitems}

\index{\_\_profilesetup() (nloed.model.Model method)@\spxentry{\_\_profilesetup()}\spxextra{nloed.model.Model method}}

\begin{fulllineitems}
\phantomsection\label{\detokenize{nloed:nloed.model.Model.__profilesetup}}\pysiglinewithargsret{\sphinxbfcode{\sphinxupquote{\_\_profilesetup}}}{\emph{\DUrole{n}{mle\_params}}, \emph{\DUrole{n}{loglik\_func}}, \emph{\DUrole{n}{fixedparams}}, \emph{\DUrole{n}{direction}}, \emph{\DUrole{n}{options}}}{}
This function creates function/solver objects for performing a profile likelihood search for the condifence boundary
in the specified direction,  the function/solver objects compute the logliklihood ratio gap
at a given radius (along the specified direction),  along with the LLR gaps 1st and 2nd derivative with respect to the radius.
marginal (free) parameters (if they exist) are optimized conditional on the fixed parameters specified by the radius and direction
the likelihood ratio gap is the negative difference between the chi\sphinxhyphen{}squared boundary and the loglik ratio at the current radius
\begin{quote}\begin{description}
\item[{Parameters}] \leavevmode\begin{itemize}
\item {} 
\sphinxstyleliteralstrong{\sphinxupquote{mle\_params}} \textendash{} mle parameter estimates,  recieved from fitting

\item {} 
\sphinxstyleliteralstrong{\sphinxupquote{loglik\_func}} \textendash{} casadi logliklihood function for the given dataset

\item {} 
\sphinxstyleliteralstrong{\sphinxupquote{fixedparams}} \textendash{} a boolean vector,  same length as the parameters,  true means cooresponding parameters fixed by direction and radius,  false values are marginal and optimized (if they exist)

\item {} 
\sphinxstyleliteralstrong{\sphinxupquote{direction}} \textendash{} a direction in parameter space,  coordinate specified as true in fixedparams are used as the search direction

\item {} 
\sphinxstyleliteralstrong{\sphinxupquote{options}} \textendash{} an options dictionary for passing user options

\end{itemize}

\item[{Returns}] \leavevmode
casadi function/ipopt solver that returns the loglik ratio gap for a given radius,  after optimizing free/marginal parameters if they exist
profile\_loglik\_jacobian\_solver: casadi function/ipopt derived derivative function that returns the derivative of the loglik ratio gap with respect to the radius (jacobian is 1x1)
profile\_loglik\_hessian\_solver: casadi function/ipopt derived 2nd derivative function that returns the 2nd derivative of the loglik ratio gap with respect to the radius (hessian is 1x1)

\item[{Return type}] \leavevmode
profile\_loglik\_solver

\end{description}\end{quote}

\end{fulllineitems}

\index{\_\_logliksearch() (nloed.model.Model method)@\spxentry{\_\_logliksearch()}\spxextra{nloed.model.Model method}}

\begin{fulllineitems}
\phantomsection\label{\detokenize{nloed:nloed.model.Model.__logliksearch}}\pysiglinewithargsret{\sphinxbfcode{\sphinxupquote{\_\_logliksearch}}}{\emph{\DUrole{n}{profile\_loglik\_solver}}, \emph{\DUrole{n}{marginal\_param}}, \emph{\DUrole{n}{options}}, \emph{\DUrole{n}{forward}\DUrole{o}{=}\DUrole{default_value}{True}}}{}
This function performs a root finding algorithm using solver\_list objects
It uses halley’s method (currently bisection as of summer 2020) to find the radius value (relative to the mle) where the loglik ratio equals the chi\sphinxhyphen{}squared level
This radius runs along the direction specified in the solver\_list when they are created
Halley’s method is a higher order extension of newton’s method for finding roots
\begin{quote}\begin{description}
\item[{Parameters}] \leavevmode\begin{itemize}
\item {} 
\sphinxstyleliteralstrong{\sphinxupquote{solver\_list}} \textendash{} solver/casadi functions for finding the loglikelihood ratio gap at a given radius from mle,  and its 1st/2nd derivatives

\item {} 
\sphinxstyleliteralstrong{\sphinxupquote{marginal\_param}} \textendash{} starting values (usually the mle) for the marginal parameters

\item {} 
\sphinxstyleliteralstrong{\sphinxupquote{options}} \textendash{} an options dictionary for passing user options

\item {} 
\sphinxstyleliteralstrong{\sphinxupquote{forward}} \textendash{} boolean,  if true search is done in the forward (positive) radius direction (relative to direction specidied in solver list),  if false perform search starting with a negative radius

\end{itemize}

\item[{Returns}] \leavevmode
returns the radius corresponding to the chi\sphinxhyphen{}squared boundary of the loglikelihood region
marginal\_param: returns the optimal setting of the marginal parameters at the boundary
ratio\_gap: returns the residual loglikelihood ratio gap at the boundary (it should be small,  within tolerance)

\item[{Return type}] \leavevmode
radius

\end{description}\end{quote}

\end{fulllineitems}

\index{\_\_creategrid() (nloed.model.Model method)@\spxentry{\_\_creategrid()}\spxextra{nloed.model.Model method}}

\begin{fulllineitems}
\phantomsection\label{\detokenize{nloed:nloed.model.Model.__creategrid}}\pysiglinewithargsret{\sphinxbfcode{\sphinxupquote{\_\_creategrid}}}{\emph{\DUrole{n}{input\_candidate\_list}}}{}
\end{fulllineitems}

\index{\_progress\_bar() (nloed.model.Model method)@\spxentry{\_progress\_bar()}\spxextra{nloed.model.Model method}}

\begin{fulllineitems}
\phantomsection\label{\detokenize{nloed:nloed.model.Model._progress_bar}}\pysiglinewithargsret{\sphinxbfcode{\sphinxupquote{\_progress\_bar}}}{\emph{\DUrole{n}{iteration}}, \emph{\DUrole{n}{total}}, \emph{\DUrole{n}{prefix}\DUrole{o}{=}\DUrole{default_value}{\textquotesingle{}\textquotesingle{}}}}{}
Helper function to print progress bar in a looped process
\begin{quote}\begin{description}
\item[{Parameters}] \leavevmode\begin{itemize}
\item {} 
\sphinxstyleliteralstrong{\sphinxupquote{iteration}} \textendash{} current iteration in the process

\item {} 
\sphinxstyleliteralstrong{\sphinxupquote{total}} \textendash{} total iterations in the process

\item {} 
\sphinxstyleliteralstrong{\sphinxupquote{prefix}} \textendash{} prefix string to name process

\end{itemize}

\end{description}\end{quote}

\end{fulllineitems}


\end{fulllineitems}



\chapter{NLoed’s Design Class}
\label{\detokenize{nloed:module-nloed.design}}\label{\detokenize{nloed:nloed-s-design-class}}\index{module@\spxentry{module}!nloed.design@\spxentry{nloed.design}}\index{nloed.design@\spxentry{nloed.design}!module@\spxentry{module}}\index{Design (class in nloed.design)@\spxentry{Design}\spxextra{class in nloed.design}}

\begin{fulllineitems}
\phantomsection\label{\detokenize{nloed:nloed.design.Design}}\pysiglinewithargsret{\sphinxbfcode{\sphinxupquote{class }}\sphinxcode{\sphinxupquote{nloed.design.}}\sphinxbfcode{\sphinxupquote{Design}}}{\emph{\DUrole{n}{models}}, \emph{\DUrole{n}{parameters}}, \emph{\DUrole{n}{objective}}, \emph{\DUrole{n}{discrete\_inputs}\DUrole{o}{=}\DUrole{default_value}{None}}, \emph{\DUrole{n}{continuous\_inputs}\DUrole{o}{=}\DUrole{default_value}{None}}, \emph{\DUrole{n}{observ\_groups}\DUrole{o}{=}\DUrole{default_value}{None}}, \emph{\DUrole{n}{fixed\_design}\DUrole{o}{=}\DUrole{default_value}{None}}, \emph{\DUrole{n}{options}\DUrole{o}{=}\DUrole{default_value}{\{\}}}}{}
Bases: \sphinxcode{\sphinxupquote{object}}

A class for experimental designs in the NLoed package
\index{round() (nloed.design.Design method)@\spxentry{round()}\spxextra{nloed.design.Design method}}

\begin{fulllineitems}
\phantomsection\label{\detokenize{nloed:nloed.design.Design.round}}\pysiglinewithargsret{\sphinxbfcode{\sphinxupquote{round}}}{\emph{\DUrole{n}{sample\_size}}, \emph{\DUrole{n}{options}\DUrole{o}{=}\DUrole{default_value}{\{\}}}}{}
This function
Args:
Return:

\end{fulllineitems}

\index{relaxed() (nloed.design.Design method)@\spxentry{relaxed()}\spxextra{nloed.design.Design method}}

\begin{fulllineitems}
\phantomsection\label{\detokenize{nloed:nloed.design.Design.relaxed}}\pysiglinewithargsret{\sphinxbfcode{\sphinxupquote{relaxed}}}{\emph{\DUrole{n}{options}\DUrole{o}{=}\DUrole{default_value}{\{\}}}}{}
This function
Args:
Return:

\end{fulllineitems}

\index{power() (nloed.design.Design method)@\spxentry{power()}\spxextra{nloed.design.Design method}}

\begin{fulllineitems}
\phantomsection\label{\detokenize{nloed:nloed.design.Design.power}}\pysiglinewithargsret{\sphinxbfcode{\sphinxupquote{power}}}{\emph{\DUrole{n}{sample\_size}}, \emph{\DUrole{n}{options}\DUrole{o}{=}\DUrole{default_value}{\{\}}}}{}
This function
Args:
Return:

\end{fulllineitems}

\index{\_optim\_settup() (nloed.design.Design method)@\spxentry{\_optim\_settup()}\spxextra{nloed.design.Design method}}

\begin{fulllineitems}
\phantomsection\label{\detokenize{nloed:nloed.design.Design._optim_settup}}\pysiglinewithargsret{\sphinxbfcode{\sphinxupquote{\_optim\_settup}}}{\emph{\DUrole{n}{fim\_list}}, \emph{\DUrole{n}{continuous\_symbol\_list}}, \emph{\DUrole{n}{continuous\_lowerbounds}}, \emph{\DUrole{n}{continuous\_upperbounds}}, \emph{\DUrole{n}{continuous\_init}}, \emph{\DUrole{n}{weight\_symbol\_list}}, \emph{\DUrole{n}{weight\_sum}}, \emph{\DUrole{n}{weight\_init}}, \emph{\DUrole{n}{options}}}{}
\end{fulllineitems}

\index{\_weighting\_settup() (nloed.design.Design method)@\spxentry{\_weighting\_settup()}\spxextra{nloed.design.Design method}}

\begin{fulllineitems}
\phantomsection\label{\detokenize{nloed:nloed.design.Design._weighting_settup}}\pysiglinewithargsret{\sphinxbfcode{\sphinxupquote{\_weighting\_settup}}}{\emph{\DUrole{n}{discrete\_input\_names}}, \emph{\DUrole{n}{discrete\_input\_grid}}, \emph{\DUrole{n}{continuous\_input\_names}}, \emph{\DUrole{n}{continuous\_symbol\_archetypes}}, \emph{\DUrole{n}{fixed\_design}}, \emph{\DUrole{n}{options}}}{}
\end{fulllineitems}

\index{\_discrete\_settup() (nloed.design.Design method)@\spxentry{\_discrete\_settup()}\spxextra{nloed.design.Design method}}

\begin{fulllineitems}
\phantomsection\label{\detokenize{nloed:nloed.design.Design._discrete_settup}}\pysiglinewithargsret{\sphinxbfcode{\sphinxupquote{\_discrete\_settup}}}{\emph{\DUrole{n}{discrete\_inputs}}, \emph{\DUrole{n}{options}}}{}
This function
Args:
Return:

\end{fulllineitems}

\index{\_continuous\_settup() (nloed.design.Design method)@\spxentry{\_continuous\_settup()}\spxextra{nloed.design.Design method}}

\begin{fulllineitems}
\phantomsection\label{\detokenize{nloed:nloed.design.Design._continuous_settup}}\pysiglinewithargsret{\sphinxbfcode{\sphinxupquote{\_continuous\_settup}}}{\emph{\DUrole{n}{continuous\_inputs}}, \emph{\DUrole{n}{options}}}{}
This function
Args:
Return:

\end{fulllineitems}

\index{\_create\_grid() (nloed.design.Design method)@\spxentry{\_create\_grid()}\spxextra{nloed.design.Design method}}

\begin{fulllineitems}
\phantomsection\label{\detokenize{nloed:nloed.design.Design._create_grid}}\pysiglinewithargsret{\sphinxbfcode{\sphinxupquote{\_create\_grid}}}{\emph{\DUrole{n}{input\_names}}, \emph{\DUrole{n}{input\_candidates}}, \emph{\DUrole{n}{constraints}}}{}
This function
Args:
Return:

\end{fulllineitems}

\index{\_sort\_inputs() (nloed.design.Design method)@\spxentry{\_sort\_inputs()}\spxextra{nloed.design.Design method}}

\begin{fulllineitems}
\phantomsection\label{\detokenize{nloed:nloed.design.Design._sort_inputs}}\pysiglinewithargsret{\sphinxbfcode{\sphinxupquote{\_sort\_inputs}}}{\emph{\DUrole{n}{newrow}}, \emph{\DUrole{n}{rows}}, \emph{\DUrole{n}{rowpntr}\DUrole{o}{=}\DUrole{default_value}{0}}, \emph{\DUrole{n}{colpntr}\DUrole{o}{=}\DUrole{default_value}{0}}}{}
\end{fulllineitems}


\end{fulllineitems}



\chapter{Index}
\label{\detokenize{nloed:index}}\begin{itemize}
\item {} 
\DUrole{xref,std,std-ref}{genindex}

\end{itemize}


\chapter{Index}
\label{\detokenize{index:index}}\begin{itemize}
\item {} 
\DUrole{xref,std,std-ref}{genindex}

\end{itemize}


\renewcommand{\indexname}{Python Module Index}
\begin{sphinxtheindex}
\let\bigletter\sphinxstyleindexlettergroup
\bigletter{n}
\item\relax\sphinxstyleindexentry{nloed.design}\sphinxstyleindexpageref{nloed:\detokenize{module-nloed.design}}
\item\relax\sphinxstyleindexentry{nloed.model}\sphinxstyleindexpageref{nloed:\detokenize{module-nloed.model}}
\end{sphinxtheindex}

\renewcommand{\indexname}{Index}
\printindex
\end{document}